\documentclass[conference,a4paper]{IEEEtran}

% Escritura mejorada de fórmulas matemáticas
\usepackage{amsmath}

% Inserción de gráficos
\usepackage{graphicx}

% Escritura de pseudocódigo
\usepackage[kw]{pseudo}

% Escritura mejorada de tablas
\usepackage{booktabs}

% Escritura mejorada de citas bibliográficas
\usepackage{cite}


% Macros traducidas
\def\contentsname{Índice general}
\def\listfigurename{Índice de figuras}
\def\listtablename{Índice de tablas}
\def\refname{Referencias}
\def\indexname{Índice alfabético}
\def\figurename{Fig.}
\def\tablename{TABLA}
\def\partname{Parte}
\def\appendixname{Apéndice}
\def\abstractname{Resumen}
% IEEE specific names
\def\IEEEkeywordsname{Palabras clave}
\def\IEEEproofname{Demostración}


\begin{document}

\title{Algoritmos genéticos para problemas de regresión no lineal.}

\author{
  \IEEEauthorblockN{Javier Santos Martín}
  \IEEEauthorblockA{
    \textit{Dpto. Ciencias de la Computación e Inteligencia Artificial}\\
    \textit{Universidad de Sevilla}\\
    Sevilla, España\\
    javier.jsm21@gmail.com | javsanmar5@alum.us.es}
  
  \and
  
  \IEEEauthorblockN{Javier Ruíz Garrido}
  \IEEEauthorblockA{
    \textit{Dpto. Ciencias de la Computación e Inteligencia Artificial}\\
    \textit{Universidad de Sevilla}\\
    Sevilla, España\\
    jrg@gmail.com | javruigar2@alum.us.es}
}

\maketitle


% Resumen
\begin{abstract}
  Escribir aquí dos párrafos indicando el objetivo principal del trabajo, y un
  resumen de las conclusiones obtenidas. Cabe mencionar que este documento se
  ha confeccionado siguiendo el formato de conferencias de IEEE (ver la guía
  para autores para más información, existen plantillas para Word y \LaTeX).
  Este documento se debe emplear como guía, se pueden añadir nuevas secciones
  según sea necesario. Es importante dotarlo de un número razonable de
  referencias bibliográficas.
\end{abstract}


% Palabras claves
\begin{IEEEkeywords}
  Inteligencia Artificial, otras palabras clave…
\end{IEEEkeywords}


\section{Introducción}

En esta sección se dedican varios párrafos para dar el contexto en el que se
desarrolla el trabajo presentado. Normalmente se va desde lo más general a lo
más preciso, para englobar el trabajo dentro de un ámbito concreto. Hay que
recordar que se debe referenciar la información que se consiga de fuentes
bibliográficas (artículos, libros, capítulos de libros, páginas web, apuntes,
etc.). Para ello se añade el ítem correspondiente en el apartado de
referencias, dotándolo de un número, y se añade la referencia al final de la
frase o párrafo correspondiente~\cite{b1}. Si se realiza en Microsoft Word o
LibreOffice, se puede hacer uso de \emph{referencias cruzadas} para actualizar
la numeración de las referencias de forma automática en todo el documento.

Después vienen un par de párrafos para dar un poco más de detalle sobre el
trabajo realizado. Hay que indicar la problemática o el objetivo que se marca,
y cómo se ha enfocado la solución propuesta en este trabajo. Finalmente, el
último párrafo se dedica a comentar la estructura del documento por secciones,
como el que sigue.


\section{Preliminares}

En esta sección se hace una breve introducción de las técnicas empleadas y
también trabajos relacionados, si los hay.


\subsection{Métodos empleados}

Describir aquí los métodos y técnicas empleadas (búsqueda en espacio de
estados, algoritmos genéticos, redes bayesianas, técnicas de clasificación,
redes neuronales, etc.). Si es necesario, separarlos en distintas subsecciones
dentro de Antecedentes.

Se pueden usar listas por puntos como sigue:
\begin{itemize}
\item Un punto: esto es un ejemplo de una lista.
\item Otro punto.
\end{itemize}

Por último, se debe hacer un uso correcto de las referencias bibliográficas,
para que el lector pueda acceder a más información~\cite{b2}. Todas las
referencias al final del documento deben ser citadas al menos una vez.


\subsection{Trabajo Relacionado}

Se puede realizar un recorrido en la literatura sobre trabajos anteriores que
estén relacionados y que sea por tanto interesante comentar aquí. Por supuesto,
añadir las referencias bibliográficas correspondientes.


\section{Metodología}

Esta sección se dedica a la descripción del método implementado en el trabajo.
Esta parte es la correspondiente a lo realmente desarrollado en el trabajo, y
se puede emplear pseudocódigo (nunca código), esquemas, tablas, etc.

\begin{figure}
  \centering
  \includegraphics{ejemplo}
  \caption{Ejemplo de un pie de figura. Imagen con derechos Creative Commons}
  \label{fig:ejemplo}
\end{figure}

A continuación, un ejemplo de uso de listas numeradas:
\begin{enumerate}
\item\label{item:dos-alumnos} \textit{Trabajos con dos alumnos:} poner nombre y
  apellidos completos de cada uno, y correos electrónicos de contacto (a ser
  posible de la Universidad de Sevilla). El orden de los alumnos se fijará por
  orden alfabético según los apellidos.
\item \textit{Trabajo con un autor:} cambiar la cabecera de la siguiente manera
  \begin{enumerate}
  \item \textit{Una sola columna:} solo se debe especificar un alumno.
  \item \textit{Información a añadir:} la misma que la especificada en el
    punto~\ref{item:dos-alumnos}.
  \end{enumerate}
\end{enumerate}

Las figuras se deben mencionar en el texto, como la
\figurename~\ref{fig:ejemplo}. También se pueden añadir ecuaciones, como la
ecuación~\eqref{eq:ejemplo}.

\begin{equation}
  \label{eq:ejemplo}
  a + b = \gamma
\end{equation}

Un ejemplo de pseudocódigo se puede observar en la
\figurename~\ref{pcd:fitness}.

\begin{figure}[t]
  \centering
  \begin{pseudo}*
    \hd{\fn{fitness}}(self, train\_data, data\_percentage) \\*
    \multicolumn{2}{l}{\textbf{Entrada}: El cromosoma actual, una lista de tuplas de} \\*
    \multicolumn{2}{l}{floats que representan los datos de entrenamiento y un} \\*
    \multicolumn{2}{l}{ float que representa el porcentaje de datos con los que} \\*
    \multicolumn{2}{l}{ será entrenado el cromosoma} \\*
    \multicolumn{2}{l}{\textbf{Salida}: Un float que representa el error cuadrático medio} \\*
    \multicolumn{2}{l}{ (RMSE)} \\
    \fn{y\_pred \(\leftarrow\) lista vacía} \\
    \fn{y\_true \(\leftarrow\) lista vacía} \\
    \fn{selected\_data \(\leftarrow\)} seleccionar aleatoriamente \\+
    \fn{(train\_data * data\_percentage) elementos de} \\+
    \fn{train\_data} \\--
    \textbf{para cada} \fn{datum} \textbf{en} \fn{selected\_data} \textbf{hacer} \\+
    agregar \fn{llamada a la funcion predict con datum} \\+ \fn{como parametro} a \fn{y\_pred} \\
    agregar \fn{último elemento de datum} a \fn{y\_true} \\-
    \fn {rmse} \(\leftarrow\) \fn{root\_mean\_squared\_error}(y\_true, y\_pred) \\
    \textbf{devolver} \fn{rmse}
  \end{pseudo}
  \caption{Función de evaluación del cromosoma (\texttt{fitness}) en pseudocódigo}
  \label{pcd:fitness}
\end{figure}


\begin{figure}[t]
  \centering
  \begin{pseudo}*
    \hd{\fn{predict}}(\textit{self, datum}) \\*
    \multicolumn{2}{l}{\textbf{Entrada}: El cromosoma actual, una tupla de floats que} \\*
    \multicolumn{2}{l}{representa una línea de los datos de entrenamiento} \\*
    \multicolumn{2}{l}{\textbf{Salida}: Un float que representa el valor predicho} \\
    \fn{prediction \(\leftarrow\) 0} \\
    \textbf{para cada} \fn{i} \textbf{en rango} \fn{len(datum) - 1} \textbf{hacer} \\+
    \textbf{si} \fn{\textit{el dato recibido i}} es menor que \fn0
    \textbf{entonces} \\+
    \fn{\textit{el exponente i}} \(\leftarrow\) 
    (redondear a entero \fn{\textit{el}}\\+
    \fn{\textit{ exponente i}} \\--
    \textbf{si no, si} \fn{\textit{el dato recibido i}} es igual que \fn0 \textbf{entonces} \\+
    \textit{\fn{el exponente i}} \(\leftarrow\) \fn{hacer el valor absoluto} \\-
    \fn{exponent \(\leftarrow\) \textit{el exponente i}} \\
    \fn{prediction} \(\mathrel{+}= \textit{\fn{el coeficiente i}} \times \fn{\textit{(el dato recibido i }} \\+
    \text{ elevado a} \fn{exponent)}\) \\--
    \fn{prediction} \(\mathrel{+}= \textit{ \fn{el último coeficiente}}\) \\
    \textbf{devolver} \fn{prediction}
  \end{pseudo}
  \caption{Función de predicción (\texttt{predict}) en pseudocódigo}
  \label{pcd:predict}
\end{figure}



\begin{figure}[t]
  \centering
  \begin{pseudo}*
    \hd{\fn{crossover}}(\textit{self}, \textit{chromosomeToCrossWith}, \textit{cross\_rate}) \\*
    \multicolumn{2}{l}{\textbf{Entrada}: El cromosoma actual, el cromosoma para cruzar y la} \\*
    \multicolumn{2}{l}{ tasa de cruce} \\*
    \multicolumn{2}{l}{\textbf{Salida}: Una lista de los dos cromosomas resultantes del cruce} \\
    \textbf{si \fn{el número random generado} es mayor que } \textit{\fn{cross\_rate}} \\+
    \textbf{entonces} \\+
    \textbf{devolver \fn{una lista con el cromosoma actual y el}} \\+
    \fn{cromosoma con el que iba a ser cruzado} \\---
    \fn{punto} \(\leftarrow\) \fn{número random entre el uno y la longitud de } \\+
    \fn{la lista de exponentes menos uno} \\---
    \fn{child1\_coefficients} \(\leftarrow\)
    \textit{\fn{los coeficientes del cromosoma actual}} \\+ 
    \textit{\fn{hasta "punto" + los coeficientes del cromosoma a cruzar}} \\+
     \textit{\fn{desde "punto"}} \\--
    \fn{child2\_coefficients} \(\leftarrow\) \textit{\fn{los coeficientes del cromosoma a cruzar}} \\+ 
    \textit{\fn{hasta "punto" + los coeficientes del cromosoma actual}} \\+
     \textit{\fn{desde "punto"}} \\--
    \fn{child1\_exponents} \(\leftarrow\)
    \textit{\fn{los exponentes del cromosoma actual}} \\+ 
    \textit{\fn{hasta "punto" + los exponentes del cromosoma a cruzar}} \\+
     \textit{\fn{desde "punto"}} \\--
    \fn{child2\_exponents} \(\leftarrow\)
    \textit{\fn{los exponentes del cromosoma a cruzar}} \\+ 
    \textit{\fn{hasta "punto" + los exponentes del cromosoma actual}} \\+
     \textit{\fn{desde "punto"}} \\--
    \textbf{devolver} \fn{lista con el cromosoma child1\_coefficients} \\+
    \fn{y child1\_exponents, y el cromosoma child2\_coefficients} \\+
    \fn{ y child2\_exponents}
  \end{pseudo}
  \caption{Función de cruza (\texttt{crossover}) en pseudocódigo}
  \label{pcd:crossover}
\end{figure}


\begin{figure}[t]
  \centering
  \begin{pseudo}*
    \hd{\fn{mutate}}(\textit{self}, \textit{mutation\_rate}, \textit{mutation\_range}) \\*
    \multicolumn{2}{l}{\textbf{Entrada}: El cromosoma actual, tasa de mutación y rango de} \\*
     \multicolumn{2}{l}{mutación} \\*
    \multicolumn{2}{l}{\textbf{Salida}: Ninguna} \\
    \textit{\fn{random\_number}} \(\leftarrow\) \fn{Número aleatorio entre 0 y 1} \\--
    \textbf{si} \textit{\fn{random\_number} es menor que}  \textit{\fn{mutation\_rate}} \textbf{entonces} \\+
        \textbf{para cada} \textit{\fn{i}} \textbf{en rango} \fn{longitud de la lista de exponentes} \\+
        \fn{ del cromosoma actual }\textbf{hacer} \\+
            \textit{\fn{coeficiente i del cromosoma actual}} \(\mathrel{+}= \fn{un valor}\\+
            \fn{aleatorio entre -mutation\_range y} \\+ \fn{mutation\_range}\) \\--
            \textit{\fn{exponente i del cromosoma actual}} \(\mathrel{+}= \fn{un valor}\\+
            \fn{aleatorio entre -mutation\_range y} \\+ \fn{mutation\_range}\) \\----
        \textit{\fn{último coeficiente del cromosoma actual}} \(\mathrel{+}= \fn{un valor}\\+
            \fn{aleatorio entre -mutation\_range y mutation\_range}\\--
  \end{pseudo}
  \caption{Función de mutación (\texttt{mutate}) en pseudocódigo}
  \label{pcd:mutate}
\end{figure}



\begin{figure}[t]
  \centering
  \begin{pseudo}*
    \hd{\fn{run}}(\textit{self}) \\*
    \multicolumn{2}{l}{\textbf{Entrada}: Instancia de la clase AG} \\*
    \multicolumn{2}{l}{\textbf{Salida}: El cromosoma con la mejor aptitud encontrada y ese} \\*
    \multicolumn{2}{l}{cromosoma sometido a la funcion test} \\
    \textit{\fn{elitism\_count}} \(\leftarrow\) 
    \fn{ratio de elitismo por el tamaño de} \\+
    \textit{\fn{población}}\\--
    \textbf{para} \textit{\fn{generation}} \textbf{en rango} \textit{\fn{máximas iteraciones}} \textbf{hacer} \\+
    \textbf{para} \textit{\fn{pair}} \textbf{en} \textit{\fn{la población actual}} \textbf{hacer} \\+
        \textit{\fn{self.population}} \(\leftarrow\) \(\text{\fn{lista de listas del primer}} \\+
        \fn{elemento de pair y de su función fitness} \\--
        \textit{\fn{self.population}} \(\leftarrow\) 
        \fn{se ordena por el segundo}\\+
        \fn{elemento de cada tupla de la lista de}\\+
        \fn{población} \\--
        \textbf{para} \textit{\fn{pair}} \textbf{en} \textit{\fn{la población actual hasta el}} \\+
        \fn{elitism\_count elemento} \textbf{hacer} \\+
        \textit{\fn{next\_generation}} \(\leftarrow\) \(\text{\fn{añadir el primer}} \\+
        \fn{elemento de pair} \\---
        \textbf{mientras} \fn{longitud de}
        \textit{\fn{next\_generation} menor que} \\+
        \textit{\fn{tamaño de población}} \textbf{hacer} \\+
            \fn{\textit{parent1}, \textit{parent2}} \(\leftarrow\) \fn{cromosoma ganador del} \\+
            \fn{\textit{torneo}, \textit{cromosoma ganador del torneo}} \\-
            \textit{\fn{offspring}} \(\leftarrow\) \textit{\fn{crossover entre parent 1 y 2}}\\
            \textbf{para cada} \textit{\fn{child}} \textbf{en} \textit{\fn{offspring}} \textbf{hacer} \\+
                \textbf{si} \fn{longitud de next\_generation} menor\\+
                que \fn{tamaño población} \\+
                    \fn{\textit{child} \(\leftarrow\) \textit{mutate cromosoma}} \\
                    \fn{\textit{next\_generation}} \(\leftarrow\) \textit{añadir}\\+ \fn{child} \\------
        \fn{\textit{best\_fitness} \(\leftarrow\) \textit{segundo elemento de la primera tupla}}\\+
        \fn{de la lista de población} \\-
        \fn{\textit{self.population} \(\leftarrow\) \textit{next\_generation}} \\-
    \textit{\fn{winner\_chromosome}} \(\leftarrow\) 
    \fn{cromosoma con mejor fitness}\\
    \textbf{devolver} 
    \fn{\textit{winner\_chromosome}, \textit{test de winner\_chromosome}} \\
  \end{pseudo}
  \caption{Función de ejecución (\texttt{run}) en pseudocódigo}
  \label{pcd:run}
\end{figure}


\begin{figure}[t]
  \centering
  \begin{pseudo}*
    \hd{\fn{tournament\_selection}}(\textit{self}, \textit{k}) \\*
    \multicolumn{2}{l}{\textbf{Entrada}: Instancia de la clase AG y número de cromosomas} \\*
    \multicolumn{2}{l}{seleccionados para el torneo (\textit{k})} \\*
    \multicolumn{2}{l}{\textbf{Salida}: Cromosoma con la mejor aptitud} \\
    \textit{\fn{tournament}} \(\leftarrow\) \fn{k elementos aleatorios de población actual}\\--
    \textit{\fn{sorted\_tournament}} \(\leftarrow\) 
        \fn{se ordena por el segundo}\\+
        \fn{elemento de cada tupla de la lista de población}\\--
    \textbf{devolver} \textit{\fn{primer elemento de sorted\_torunament}} \\
  \end{pseudo}
  \caption{Selección de torneo en pseudocódigo}
  \label{pcd:tournament_selection}
\end{figure}




\begin{figure}[t]
  \centering
  \begin{pseudo}*
    \hd{\fn{test}}(\textit{self}, \textit{chromosome}) \\*
    \multicolumn{2}{l}{\textbf{Entrada}: Instancia de la clase AG y cromosoma a probar} \\*
    \multicolumn{2}{l}{\textbf{Salida}: Lista de valores predichos} \\
    \textit{\fn{y\_pred\(\leftarrow\) lista vacía}} \\--
    \textbf{para cada} \textit{\fn{datum}} \textbf{en} \textit{\fn{test\_data}} \textbf{hacer} \\+
        \textit{\fn{predicted \(\leftarrow\) \textit{predict datum}}} \\
        \textit{\fn{y\_pred \(\leftarrow\) \textit{añadir predicted}}} \\--
    \textbf{devolver} \textit{\fn{y\_pred}} \\
  \end{pseudo}
  \caption{Función de prueba (\texttt{test}) en pseudocódigo}
  \label{pcd:test}
\end{figure}










\section{Resultados}

En esta sección se detallarán tanto los experimentos realizados como los
resultados conseguidos:
\begin{itemize}
\item Los experimentos realizados, indicando razonadamente la configuración
  empleada, qué se quiere determinar, y como se ha medido.
\item Los resultados obtenidos en cada experimento, explicando en cada caso lo
  que se ha conseguido.
\item Análisis de los resultados, haciendo comparativas y obteniendo
  conclusiones.
\end{itemize}

Se pueden hacer uso de tablas, como el ejemplo de la tabla~\ref{tab:ejemplo}.

\begin{table}
  \caption{Ejemplo de tabla}
  \label{tab:ejemplo}
  \centering
  \begin{tabular}{ccc}
    \toprule
    A & B & C \\
    \midrule
    1 & 2 & 3 \\
    4 & 5 & 6 \\
    \bottomrule
  \end{tabular}
\end{table}


\section{Conclusiones}

Finalmente, se dedica la última sección para indicar las conclusiones obtenidas
del trabajo. Se puede dedicar un párrafo para realizar un resumen sucinto del
trabajo, con los experimentos y resultados. Seguidamente, uno o dos párrafos
con conclusiones. Se suele dedicar un párrafo final con ideas de mejora y
trabajo futuro.


\begin{thebibliography}{00}
\bibitem{b1} G. Eason, B. Noble, and I. N. Sneddon, ``On certain integrals of Lipschitz-Hankel type involving products of Bessel functions,'' Phil. Trans. Roy. Soc. London, vol. A247, pp. 529--551, April 1955.
\bibitem{b2} J. Clerk Maxwell, A Treatise on Electricity and Magnetism, 3rd ed., vol. 2. Oxford: Clarendon, 1892, pp.68--73.
\bibitem{b3} I. S. Jacobs and C. P. Bean, ``Fine particles, thin films and exchange anisotropy,'' in Magnetism, vol. III, G. T. Rado and H. Suhl, Eds. New York: Academic, 1963, pp. 271--350.
\bibitem{b4} K. Elissa, ``Title of paper if known,'' unpublished.
\bibitem{b5} R. Nicole, ``Title of paper with only first word capitalized,'' J. Name Stand. Abbrev., in press.
\bibitem{b6} Y. Yorozu, M. Hirano, K. Oka, and Y. Tagawa, ``Electron spectroscopy studies on magneto-optical media and plastic substrate interface,'' IEEE Transl. J. Magn. Japan, vol. 2, pp. 740--741, August 1987 [Digests 9th Annual Conf. Magnetics Japan, p. 301, 1982].
\bibitem{b7} M. Young, The Technical Writer's Handbook. Mill Valley, CA: University Science, 1989.
\end{thebibliography}




\end{document}
