\documentclass[conference,a4paper]{IEEEtran}

% Escritura mejorada de fórmulas matemáticas
\usepackage{amsmath}

% Inserción de gráficos
\usepackage{graphicx}

% Escritura de pseudocódigo
\usepackage[kw]{pseudo}

% Escritura mejorada de tablas
\usepackage{booktabs}

% Escritura mejorada de citas bibliográficas
\usepackage{cite}


% Macros traducidas
\def\contentsname{Índice general}
\def\listfigurename{Índice de figuras}
\def\listtablename{Índice de tablas}
\def\refname{Referencias}
\def\indexname{Índice alfabético}
\def\figurename{Fig.}
\def\tablename{TABLA}
\def\partname{Parte}
\def\appendixname{Apéndice}
\def\abstractname{Resumen}
% IEEE specific names
\def\IEEEkeywordsname{Palabras clave}
\def\IEEEproofname{Demostración}


\begin{document}

\title{Algoritmos genéticos para problemas de regresión no lineal.}

\author{
  \IEEEauthorblockN{Javier Santos Martín}
  \IEEEauthorblockA{
    \textit{Dpto. Ciencias de la Computación e Inteligencia Artificial}\\
    \textit{Universidad de Sevilla}\\
    Sevilla, España\\
    javier.jsm21@gmail.com | javsanmar5@alum.us.es}
  
  \and
  
  \IEEEauthorblockN{Javier Ruíz Garrido}
  \IEEEauthorblockA{
    \textit{Dpto. Ciencias de la Computación e Inteligencia Artificial}\\
    \textit{Universidad de Sevilla}\\
    Sevilla, España\\
    jjrg@gmail.com | javruigar2@alum.us.es}
}

\maketitle


% Resumen
\begin{abstract}
  Escribir aquí dos párrafos indicando el objetivo principal del trabajo, y un
  resumen de las conclusiones obtenidas. Cabe mencionar que este documento se
  ha confeccionado siguiendo el formato de conferencias de IEEE (ver la guía
  para autores para más información, existen plantillas para Word y \LaTeX).
  Este documento se debe emplear como guía, se pueden añadir nuevas secciones
  según sea necesario. Es importante dotarlo de un número razonable de
  referencias bibliográficas.
\end{abstract}


% Palabras claves
\begin{IEEEkeywords}
  Inteligencia Artificial, otras palabras clave…
\end{IEEEkeywords}


\section{Introducción}

En esta sección se dedican varios párrafos para dar el contexto en el que se
desarrolla el trabajo presentado. Normalmente se va desde lo más general a lo
más preciso, para englobar el trabajo dentro de un ámbito concreto. Hay que
recordar que se debe referenciar la información que se consiga de fuentes
bibliográficas (artículos, libros, capítulos de libros, páginas web, apuntes,
etc.). Para ello se añade el ítem correspondiente en el apartado de
referencias, dotándolo de un número, y se añade la referencia al final de la
frase o párrafo correspondiente~\cite{b1}. Si se realiza en Microsoft Word o
LibreOffice, se puede hacer uso de \emph{referencias cruzadas} para actualizar
la numeración de las referencias de forma automática en todo el documento.

Después vienen un par de párrafos para dar un poco más de detalle sobre el
trabajo realizado. Hay que indicar la problemática o el objetivo que se marca,
y cómo se ha enfocado la solución propuesta en este trabajo. Finalmente, el
último párrafo se dedica a comentar la estructura del documento por secciones,
como el que sigue.


\section{Preliminares}

La idea principal de este trabajo es la aplicación de algoritmos genéticos para abordar problemas de regresión no lineal. La representación de los cromosomas se realiza con el propósito de modelar posibles soluciones a este problema. Dado que se trata de un problema de regresión, la estructura del cromosoma incluye tanto los coeficientes como los exponentes. Para ello, el cromosoma se organiza en dos listas: una de coeficientes y otra de exponentes correspondientes a las variables del problema. La evaluación de estos cromosomas se lleva a cabo utilizando el error cuadrático medio (RMSE) entre los valores reales esperados y los valores predichos generados por cada cromosoma.

A continuación, se detallan las diferentes partes de este algoritmo:

\subsection{Cromosoma}

\textbf{Propiedades:}
\begin{itemize}
    \item \textbf{Lista de coeficientes:} Esta lista contiene \( n + 1 \) elementos de tipo \texttt{float}, donde \( n \) es el número de variables independientes. El elemento \( i \) de la lista representa el coeficiente de la variable \( x_i \), mientras que el último elemento representa el término independiente.
    \item \textbf{Lista de exponentes:} Esta lista contiene \( n \) elementos de tipo \texttt{float}, donde \( n \) es el número de variables independientes. El elemento \( i \) de la lista representa el exponente de la variable \( x_i \).    
\end{itemize}

Aunque, a priori, todos los elementos de ambas listas no tienen restricciones en cuanto a los valores que pueden adoptar, existen dos situaciones que podrían limitar estos valores:

1. \textbf{La variable \( x_i \) es negativa en alguno de los puntos aportados como datos de entrenamiento}: En este caso, el exponente de la función no podría adoptar valores fraccionarios con denominador par, de la forma \( k/m \) tal que \( m \) sea par, en el contexto de los números reales. Sin embargo, esto presenta un problema en Python, ya que cualquier número negativo elevado a una potencia fraccionaria, sea esta par o impar, genera un número complejo, lo cual no es deseable porque estamos trabajando en R.

2. \textbf{La variable \( x_i \) es cero en alguno de los puntos aportados como datos de entrenamiento}: Aquí, el problema surge con los exponentes negativos, pues esta operación está indeterminada. Por tanto, para resolver este problema, utilizamos el valor absoluto del exponente, ya que al saber que la función a interpolar está definida en \( x_i = 0 \), podemos afirmar que \( e_i > 0 \).

Utilizando los valores de este cromosoma, obtendríamos una \(\hat{y}\) de la siguiente forma:
\[
\hat{y} = \sum_{i=0}^{n} c_i x_i^{e_i} + c_{n+1}
\]




\subsubsection{Fitness}




Por último, se debe hacer un uso correcto de las referencias bibliográficas,
para que el lector pueda acceder a más información~\cite{b2}. Todas las
referencias al final del documento deben ser citadas al menos una vez.


\subsection{Trabajo Relacionado}

Se puede realizar un recorrido en la literatura sobre trabajos anteriores que
estén relacionados y que sea por tanto interesante comentar aquí. Por supuesto,
añadir las referencias bibliográficas correspondientes.


\section{Metodología}

Esta sección se dedica a la descripción del método implementado en el trabajo.
Esta parte es la correspondiente a lo realmente desarrollado en el trabajo, y
se puede emplear pseudocódigo (nunca código), esquemas, tablas, etc.

\begin{figure}
  \centering
  \includegraphics{ejemplo}
  \caption{Ejemplo de un pie de figura. Imagen con derechos Creative Commons}
  \label{fig:ejemplo}
\end{figure}

A continuación, un ejemplo de uso de listas numeradas:
\begin{enumerate}
\item\label{item:dos-alumnos} \textit{Trabajos con dos alumnos:} poner nombre y
  apellidos completos de cada uno, y correos electrónicos de contacto (a ser
  posible de la Universidad de Sevilla). El orden de los alumnos se fijará por
  orden alfabético según los apellidos.
\item \textit{Trabajo con un autor:} cambiar la cabecera de la siguiente manera
  \begin{enumerate}
  \item \textit{Una sola columna:} solo se debe especificar un alumno.
  \item \textit{Información a añadir:} la misma que la especificada en el
    punto~\ref{item:dos-alumnos}.
  \end{enumerate}
\end{enumerate}

Las figuras se deben mencionar en el texto, como la
\figurename~\ref{fig:ejemplo}. También se pueden añadir ecuaciones, como la
ecuación~\eqref{eq:ejemplo}.

\begin{equation}
  \label{eq:ejemplo}
  a + b = \gamma
\end{equation}

Un ejemplo de pseudocódigo se puede observar en la
\figurename~\ref{pcd:mergesort}.

\begin{figure}
  \begin{pseudo}*
    \hd{\fn{mergesort}}(V) \\*
    \multicolumn{2}{l}{\textbf{Entrada}: un vector \( V \)} \\*
    \multicolumn{2}{l}{\textbf{Salida}: un vector con los elementos de \( V \)
      en orden} \\
    si \( V \) \textnormal{es unitario} entonces \\+
    devolver \( V \) \\-
    si no entonces \\+
    \( V_{1} \leftarrow \textnormal{primera mitad de } V\) \\
    \( V_{2} \leftarrow \textnormal{segunda mitad de } V\) \\
    \( V_{1} \leftarrow \pr{mergesort}(V_{1}) \) \\
    \( V_{2} \leftarrow \pr{mergesort}(V_{2}) \) \\
    devolver \fn{mezcla}(V_{1}, V_{2})
  \end{pseudo}
  
  \begin{pseudo}*
    \hd{\fn{mezcla}}(V_{1}, V_{2}) \\*
    \multicolumn{2}{l}{%
      \textbf{Entrada}: dos vectores \( V_{1} \) y \( V_{2} \) ordenados
    } \\*
    \multicolumn{2}{l}{%
      \textbf{Salida}: un vector con los elementos de \( V_{1} \) y \( V_{2} \)
      en orden
    } \\
    si \( V_{1} \) \textnormal{no tiene elementos} entonces \\+
    devolver \( V_{2} \) \\-
    si no si \( V_{2} \) \textnormal{no tiene elementos} entonces \\+
    devolver \( V_{1} \) \\-
    si no entonces \\+
    \( x_{1} \leftarrow \textnormal{primer elemento de } V_{1} \) \\
    \( x_{2} \leftarrow \textnormal{primer elemento de } V_{2} \) \\
    si \( x_{1} \leq x_{2} \) entonces \\+
    \( x \leftarrow x_{1} \) \\
    \textnormal{quitar el primer elemento de} \( V_{1} \) \\-
    si no entonces \\+
    \( x \leftarrow x_{2} \) \\
    \textnormal{quitar el primer elemento de} \( V_{2} \) \\-
    \( V \leftarrow \fn{mezcla}(V_{1}, V_{2}) \) \\
    \textnormal{añadir} \( x \)
    \textnormal{como primer elemento de} \( V \) \\
    devolver \( V \)
  \end{pseudo}
  \caption{Algoritmo de ordenación \texttt{MergeSort}}
  \label{pcd:mergesort}
\end{figure}


\section{Resultados}

En esta sección se detallarán tanto los experimentos realizados como los
resultados conseguidos:
\begin{itemize}
\item Los experimentos realizados, indicando razonadamente la configuración
  empleada, qué se quiere determinar, y como se ha medido.
\item Los resultados obtenidos en cada experimento, explicando en cada caso lo
  que se ha conseguido.
\item Análisis de los resultados, haciendo comparativas y obteniendo
  conclusiones.
\end{itemize}

Se pueden hacer uso de tablas, como el ejemplo de la tabla~\ref{tab:ejemplo}.

\begin{table}
  \caption{Ejemplo de tabla}
  \label{tab:ejemplo}
  \centering
  \begin{tabular}{ccc}
    \toprule
    A & B & C \\
    \midrule
    1 & 2 & 3 \\
    4 & 5 & 6 \\
    \bottomrule
  \end{tabular}
\end{table}


\section{Conclusiones}

Finalmente, se dedica la última sección para indicar las conclusiones obtenidas
del trabajo. Se puede dedicar un párrafo para realizar un resumen sucinto del
trabajo, con los experimentos y resultados. Seguidamente, uno o dos párrafos
con conclusiones. Se suele dedicar un párrafo final con ideas de mejora y
trabajo futuro.


\begin{thebibliography}{00}
\bibitem{b1} G. Eason, B. Noble, and I. N. Sneddon, ``On certain integrals of Lipschitz-Hankel type involving products of Bessel functions,'' Phil. Trans. Roy. Soc. London, vol. A247, pp. 529--551, April 1955.
\bibitem{b2} J. Clerk Maxwell, A Treatise on Electricity and Magnetism, 3rd ed., vol. 2. Oxford: Clarendon, 1892, pp.68--73.
\bibitem{b3} I. S. Jacobs and C. P. Bean, ``Fine particles, thin films and exchange anisotropy,'' in Magnetism, vol. III, G. T. Rado and H. Suhl, Eds. New York: Academic, 1963, pp. 271--350.
\bibitem{b4} K. Elissa, ``Title of paper if known,'' unpublished.
\bibitem{b5} R. Nicole, ``Title of paper with only first word capitalized,'' J. Name Stand. Abbrev., in press.
\bibitem{b6} Y. Yorozu, M. Hirano, K. Oka, and Y. Tagawa, ``Electron spectroscopy studies on magneto-optical media and plastic substrate interface,'' IEEE Transl. J. Magn. Japan, vol. 2, pp. 740--741, August 1987 [Digests 9th Annual Conf. Magnetics Japan, p. 301, 1982].
\bibitem{b7} M. Young, The Technical Writer's Handbook. Mill Valley, CA: University Science, 1989.
\end{thebibliography}


\end{document}
